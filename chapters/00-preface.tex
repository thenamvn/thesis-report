\chapter*{Abstract}
\addcontentsline{toc}{chapter}{Abstract}
The rapid digital transformation of the global education sector has created an urgent need for e-learning platforms that are not only informative but also engaging, scalable, and secure. Traditional online learning systems frequently struggle with low student motivation, static content delivery, and significant challenges in upholding academic integrity during remote assessments. This thesis presents the design, implementation, and comprehensive evaluation of TEKUTOKO, a novel web-based educational platform engineered to address these critical shortcomings through the synergistic integration of gamification, artificial intelligence, and a distributed microservice-based architecture.

The TEKUTOKO system provides a rich, interactive environment where educators can design and host "game rooms" or secure "test rooms". A cornerstone innovation of this platform is the AI Question Generator, a robust module that leverages Google's advanced Gemini Pro API to automate the creation of diverse and contextually relevant quiz questions from topics or existing documents, thereby significantly reducing the content creation burden for educators.

To foster sustained user motivation, TEKUTOKO incorporates a comprehensive gamification and reward mechanism, empowering hosts to issue digital vouchers which can be verified via QR codes. A location-based discovery feature utilizes GPS to help users find nearby educational events, promoting community engagement. To safeguard the credibility of online assessments, a non-invasive, browser-based proctoring system is implemented to monitor and flag suspicious user behaviors, such as tab switching and prolonged inactivity. The platform is architected upon a modern, high-performance technology stack, featuring a React.js frontend, a core Node.js backend, a specialized Python microservice for document processing, a MySQL database, and Firebase for storage, ensuring scalability and maintainability.

This research rigorously evaluates the TEKUTOKO platform through a series of controlled experiments measuring system performance, user engagement via the System Usability Scale (SUS), educator efficiency, and the efficacy of the anti-cheating system. The findings conclusively demonstrate that the strategic integration of these technologies leads to a statistically significant improvement in user satisfaction, engagement, and assessment integrity. This thesis contributes a comprehensive, evidence-based framework for developing next-generation e-learning systems that are interactive, intelligent, and secure.

\vspace{1cm}
\noindent\textbf{Keywords:} Gamification, E-Learning, AI Content Generation, Online Proctoring, Anti-Cheating Systems, Microservices, Educational Technology, System Architecture, React.js, Node.js, User Engagement.

\cleardoublepage

\chapter*{Acknowledgment}
\addcontentsline{toc}{chapter}{Acknowledgment}
I wish to express my most profound and sincere gratitude to my academic supervisor, \textbf{Dr. Nguyen Van Tinh}, for their invaluable guidance, unwavering support, and insightful mentorship throughout the duration of this research project. Their expertise in software architecture and academic research was instrumental in shaping the direction of this thesis and navigating the complexities of system design and implementation. Their constructive feedback and constant encouragement were a source of motivation from the initial proposal to the final manuscript.

I extend my heartfelt thanks to the faculty members and administrative staff of the \textbf{Faculty of Information Technology at Vietnam Japan University}. Their dedication to providing a high-quality educational environment and their willingness to share their knowledge provided me with the foundational skills and resources necessary to undertake this ambitious project.

I am also deeply grateful to my family and friends for their unending support, patience, and belief in me. Their encouragement during the challenging phases of this work was a constant source of strength.

Finally, I would like to acknowledge the participants of the user study, whose time and thoughtful feedback were crucial for the evaluation and refinement of the TEKUTOKO platform.