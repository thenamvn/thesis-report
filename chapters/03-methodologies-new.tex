\chapter{METHODOLOGIES}
\label{chap:methodologies}

This chapter details the comprehensive methodologies employed in the design, development, and implementation of the TEKUTOKO platform. It encompasses three major components: requirements engineering and system analysis, architectural design and database modeling, and technical implementation. Each section describes the approaches, tools, and techniques used to transform the research objectives into a functional, scalable, and effective e-learning system.

\section{Requirements Engineering}
\label{sec:requirements-engineering}

The requirements for the TEKUTOKO platform were gathered and refined through a multi-faceted approach to ensure comprehensive coverage of user needs and technical feasibility.

\subsection{Requirements Gathering Methodology}

\begin{enumerate}
    \item \textbf{Literature Review Analysis:} The comprehensive analysis of academic papers on e-learning, gamification, and online proctoring (as detailed in Chapter \ref{chap:lit-review}) helped identify established best practices and core pedagogical needs. This provided a theoretical foundation for feature selection and priority determination.
    
    \item \textbf{Competitive Analysis:} A thorough examination of existing platforms (Kahoot!, Quizlet, Google Classroom, and specialized proctoring tools) was conducted to determine market standards, identify common features, and pinpoint opportunities for innovation. Feature matrices were developed to systematically compare capabilities across platforms.
    
    \item \textbf{Stakeholder Needs Analysis:} Potential end-users, including educators and students, were consulted to understand their primary pain points with current systems. Key desired features included faster content creation, more engaging activities, and a trustworthy but non-invasive method for conducting online tests. Semi-structured interviews were conducted with 10 educators and 15 students to gather qualitative insights.
    
    \item \textbf{Prototyping and Feedback:} Low-fidelity mockups and user flow diagrams were created to visualize core concepts and gather early feedback, ensuring the proposed system would be intuitive and user-friendly. Iterative feedback sessions helped refine interface designs and interaction patterns.
\end{enumerate}

\subsection{Functional Requirements}

Functional requirements describe the specific behaviors, features, and functions the system must perform. They are detailed in Table \ref{tab:func-req}.

\renewcommand{\arraystretch}{1.5}
\begin{longtable}{l l p{9cm}}
\caption{Functional Requirements} \label{tab:func-req} \\
\toprule
\textbf{ID} & \textbf{Category} & \textbf{Requirement Description} \\
\midrule
\endfirsthead
\multicolumn{3}{c}{{\bfseries Table \thetable\ continued from previous page}} \\
\toprule
\textbf{ID} & \textbf{Category} & \textbf{Requirement Description} \\
\midrule
\endhead
\bottomrule
\endfoot
\bottomrule
\endlastfoot

% FR1: User & Profile Management
\multicolumn{3}{l}{\textbf{FR1: User \& Profile Management}} \\
\midrule
FR1.1 & User Authentication & Users must be able to register, log in, and log out using an email/password combination. The system must use JWT for session management. \\
FR1.2 & Profile System & Each user must have a profile page displaying their information, created rooms, and followers/following count. \\
FR1.3 & Social Interaction & Users must be able to follow other users (hosts) and share links to user profiles. \\
\midrule

% FR2: Room Creation & Management (Host)
\multicolumn{3}{l}{\textbf{FR2: Room Creation \& Management (Host)}} \\
\midrule
FR2.1 & Room Type Selection & Hosts must be able to create two distinct types of rooms: a gamified "Quiz Room" or a secure "Test Room". \\
FR2.2 & Content Creation & Hosts must be able to add questions manually (multiple-choice, text input, file upload) or import them. \\
FR2.3 & AI Question Generation & The system must provide an interface for hosts to generate questions automatically by specifying a topic and parameters. \\
FR2.4 & DOCX Import & The system must allow hosts to upload a `.docx` file, which is then parsed by a microservice to populate questions and answers for a room. \\
FR2.5 & Room Configuration & Hosts must be able to set room titles, descriptions, rules, and configure GPS location settings for discovery. \\
FR2.6 & Results Dashboard & Hosts must have access to a dashboard to view participant scores, submitted answers, and proctoring logs for their rooms. \\
\midrule

% FR3: Participant Interaction
\multicolumn{3}{l}{\textbf{FR3: Participant Interaction}} \\
\midrule
FR3.1 & Room Participation & Users must be able to join a room using a unique code or by clicking a direct link. \\
FR3.2 & Answering Questions & Participants must be able to view questions and submit their answers within the defined format for each room type. \\
FR3.3 & Real-time Updates & In Quiz Rooms, the leaderboard and scores must update in near real-time. \\
\midrule

% FR4: Gamification & Rewards
\multicolumn{3}{l}{\textbf{FR4: Gamification \& Rewards}} \\
\midrule
FR4.1 & Leaderboard & Quiz Rooms must feature a live leaderboard displaying participant scores and rankings. \\
FR4.2 & Reward Configuration & Hosts must be able to create digital rewards (vouchers, tickets) for completing a Quiz Room. \\
FR4.3 & Reward Issuance & The system must automatically issue a unique QR code for the reward to participants who meet the completion criteria. \\
FR4.4 & QR Code Verification & The system must provide a mechanism for hosts to scan a participant's QR code to validate and redeem the reward, marking it as used. \\
\midrule

% FR5: Anti-Cheating / Proctoring (Test Room)
\multicolumn{3}{l}{\textbf{FR5: Anti-Cheating / Proctoring (Test Room)}} \\
\midrule
FR5.1 & Tab-Switching Detection & The system must detect when a participant navigates away from the test tab/window and log this event. \\
FR5.2 & Inactivity Monitoring & The system must monitor for prolonged periods of user inactivity during a test and flag it as a potential issue. \\
FR5.3 & Proctoring Logs & All detected suspicious events must be logged with timestamps and made available to the host. \\
\midrule

% FR6: Discovery & Community
\multicolumn{3}{l}{\textbf{FR6: Discovery \& Community}} \\
\midrule
FR6.1 & Room Discovery & The system must provide a discovery page with search functionality and a "Nearby" feature that uses the device's GPS to find local rooms. \\
FR6.2 & Popularity Ranking & The discovery page should feature and suggest rooms based on popularity metrics (e.g., number of participants, host followers). \\
\midrule

% FR7: Administrator Panel
\multicolumn{3}{l}{\textbf{FR7: Administrator Panel}} \\
\midrule
FR7.1 & System Dashboard & An admin panel must exist to display system-wide statistics (e.g., total users, active rooms). \\
FR7.2 & Management & Administrators must be able to manage users and rooms (e.g., delete inappropriate content, resolve user reports). \\

\end{longtable}

\subsection{Non-Functional Requirements}

Non-functional requirements define the quality attributes, performance standards, and constraints of the system, detailed in Table \ref{tab:non-func-req}.

\renewcommand{\arraystretch}{1.5}
\begin{longtable}{l l p{9cm}}
\caption{Non-Functional Requirements} \label{tab:non-func-req} \\
\toprule
\textbf{ID} & \textbf{Category} & \textbf{Requirement Description} \\
\midrule
\endfirsthead
\multicolumn{3}{c}{{\bfseries Table \thetable\ continued from previous page}} \\
\toprule
\textbf{ID} & \textbf{Category} & \textbf{Requirement Description} \\
\midrule
\endhead
\bottomrule
\endfoot
\bottomrule
\endlastfoot

% NFR1: Performance
\multicolumn{3}{l}{\textbf{NFR1: Performance}} \\
\midrule
NFR1.1 & API Response Time & 95\% of API requests under normal load should complete in under 500ms. AI generation requests are exempt but should not exceed 20 seconds. \\
NFR1.2 & Concurrency & The system must support at least 100 concurrent users in a single real-time quiz room without significant performance degradation. \\
\midrule

% NFR2: Scalability
\multicolumn{3}{l}{\textbf{NFR2: Scalability}} \\
\midrule
NFR2.1 & Horizontal Scaling & The backend architecture must be stateless and allow for horizontal scaling by adding more server instances behind a load balancer. \\
NFR2.2 & Microservice Independence & The Python microservice must be independently scalable from the main Node.js backend. \\
\midrule

% NFR3: Usability
\multicolumn{3}{l}{\textbf{NFR3: Usability}} \\
\midrule
NFR3.1 & User Interface & The UI must be intuitive, responsive across devices (desktop, tablet, mobile), and adhere to modern design principles. \\
NFR3.2 & Accessibility & The platform should follow Web Content Accessibility Guidelines (WCAG) 2.1 Level A standards. \\
\midrule

% NFR4: Security
\multicolumn{3}{l}{\textbf{NFR4: Security}} \\
\midrule
NFR4.1 & Data Transmission & All communication between the client and server must be encrypted using HTTPS/TLS. \\
NFR4.2 & Authentication & User authentication must be secured using JSON Web Tokens (JWT). Passwords must be securely hashed (e.g., using bcrypt) before storage. \\
NFR4.3 & Input Validation & The system must validate and sanitize all user inputs on both the client and server sides to prevent XSS, SQL Injection, and other vulnerabilities. \\
\midrule

% NFR5: Reliability & Maintainability
\multicolumn{3}{l}{\textbf{NFR5: Reliability \& Maintainability}} \\
\midrule
NFR5.1 & Uptime & The system should maintain a service uptime of at least 99.5\%. \\
NFR5.2 & Modularity & The codebase must be well-documented and organized into logical, loosely coupled modules to facilitate maintenance and future development. \\

\end{longtable}

\FloatBarrier

\section{System Design and Architecture}
\label{sec:system-architecture}

This section provides a detailed technical blueprint of the TEKUTOKO platform, outlining the high-level architectural paradigm, describing the individual components and their interactions, and detailing the database schema and API design.

\subsection{High-Level System Architecture}

The TEKUTOKO platform is engineered using a \textbf{microservice-based architecture} to promote modularity, independent scalability, and technological flexibility. The architecture decouples the client-facing presentation layer from the backend business logic and specialized processing services. This separation of concerns is critical for building a resilient and maintainable system.

The main components are the React frontend, a core Node.js backend API, a specialized Python microservice for document processing, a relational database, and integrations with third-party cloud services for file storage and AI capabilities. The overall architecture is depicted in Figure \ref{fig:system-architecture}.

\begin{figure}[htbp]
\centering
\includegraphics[width=\textwidth]{figures/system-architecture.png}
\caption{High-Level System Architecture of TEKUTOKO}
\label{fig:system-architecture}
\end{figure}

\paragraph{Architectural Flow:}
\begin{enumerate}
    \item The user interacts with the \textbf{React Single-Page Application (SPA)} in their browser.
    \item All API requests are sent via HTTPS to a central \textbf{API Gateway}, which acts as a reverse proxy and load balancer.
    \item Standard requests (user auth, room management, etc.) are routed to the \textbf{Node.js Backend}, the system's core. Requests specifically for DOCX file parsing are routed to the dedicated \textbf{Python Microservice}.
    \item The Node.js backend handles all business logic, interacting with the \textbf{MySQL Database} for persistent data storage.
    \item For file uploads, the Node.js backend generates a secure, short-lived "signed URL" from \textbf{Firebase Storage} and sends it to the client.
    \item The client uses this signed URL to upload the file directly to Firebase Storage, offloading bandwidth from the backend server.
    \item The backend orchestrates calls to the external \textbf{Google Gemini API} for question generation, handling prompt engineering and response parsing.
\end{enumerate}

\subsection{Component and Module Design}

\begin{itemize}
    \item \textbf{Frontend (React.js):} A modern SPA built with React 18 and styled with Tailwind CSS. It is responsible for rendering the entire user interface and managing client-side state using React Hooks and Context API. It communicates with the backend services via REST APIs.

    \item \textbf{Backend (Node.js/Express):} The central nervous system of the platform. Its responsibilities include providing RESTful API endpoints, managing JWT-based authentication, handling business logic for all core features, and orchestrating calls to other services.

    \item \textbf{Python Microservice (FastAPI/Flask):} A specialized service responsible for the complex task of processing `.docx` files. Its sole purpose is to accept a document, extract questions, options, and images, and convert them into a structured JSON format ready for serving. It orchestrates external command-line tools like \textbf{Pandoc} for document structure conversion and \textbf{ImageMagick} for image format transcoding (e.g., WMF/EMF to WebP). The processed output is stored in a static directory, served via HTTP, and its metadata is recorded in the database.

    \item \textbf{MySQL Database:} A relational database for all structured, persistent data. This includes user accounts, room configurations, questions, submissions, reward vouchers, and proctoring logs. The relational model ensures data integrity.

    \item \textbf{Firebase Storage:} A cloud-based object storage service used for all binary files, such as user avatars, room cover images, uploaded `.docx` documents, and files submitted by participants as answers.
\end{itemize}

\subsection{Database Design}

The relationships between the main entities in the system are illustrated in the ERD in Figure \ref{fig:erd-diagram}.

\begin{figure}[htbp]
\centering
\includegraphics[width=\textwidth]{figures/erd-diagram.png}
\caption{Database Entity-Relationship Diagram (ERD)}
\label{fig:erd-diagram}
\end{figure}

Below are the SQL schemas for some of the key tables in the database.

\begin{lstlisting}[language=SQL, caption={SQL Schema for the `users` and `rooms` tables}]
-- Users table
CREATE TABLE `users` (
  `user_id` INT AUTO_INCREMENT PRIMARY KEY,
  `username` VARCHAR(50) NOT NULL UNIQUE,
  `email` VARCHAR(100) NOT NULL UNIQUE,
  `password_hash` VARCHAR(255) NOT NULL,
  `created_at` TIMESTAMP DEFAULT CURRENT_TIMESTAMP
);

-- Rooms table
CREATE TABLE `rooms` (
  `room_id` INT AUTO_INCREMENT PRIMARY KEY,
  `host_id` INT NOT NULL,
  `room_code` VARCHAR(8) NOT NULL UNIQUE,
  `title` VARCHAR(255) NOT NULL,
  `room_type` ENUM('quiz', 'test') NOT NULL,
  `gps_lat` DECIMAL(10, 8),
  `gps_lng` DECIMAL(11, 8),
  `created_at` TIMESTAMP DEFAULT CURRENT_TIMESTAMP,
  FOREIGN KEY (`host_id`) REFERENCES `users`(`user_id`) ON DELETE CASCADE
);
\end{lstlisting}

\begin{lstlisting}[language=SQL, caption={SQL Schema for the `proctoring_logs` and `vouchers` tables}]
-- Proctoring logs table
CREATE TABLE `proctoring_logs` (
  `log_id` INT AUTO_INCREMENT PRIMARY KEY,
  `room_id` INT NOT NULL,
  `user_id` INT NOT NULL,
  `event_type` ENUM('tab_switch', 'inactivity', 'paste_attempt') NOT NULL,
  `details` TEXT,
  `event_timestamp` TIMESTAMP DEFAULT CURRENT_TIMESTAMP,
  FOREIGN KEY (`room_id`) REFERENCES `rooms`(`room_id`) ON DELETE CASCADE,
  FOREIGN KEY (`user_id`) REFERENCES `users`(`user_id`) ON DELETE CASCADE
);

-- Vouchers table
CREATE TABLE `vouchers` (
  `voucher_id` INT AUTO_INCREMENT PRIMARY KEY,
  `room_id` INT NOT NULL,
  `user_id` INT NOT NULL,
  `qr_code_data` VARCHAR(255) NOT NULL UNIQUE,
  `reward_description` TEXT NOT NULL,
  `is_redeemed` BOOLEAN NOT NULL DEFAULT FALSE,
  `issued_at` TIMESTAMP DEFAULT CURRENT_TIMESTAMP,
  `redeemed_at` TIMESTAMP NULL,
  FOREIGN KEY (`room_id`) REFERENCES `rooms`(`room_id`),
  FOREIGN KEY (`user_id`) REFERENCES `users`(`user_id`)
);
\end{lstlisting}

\subsection{API Design and Specification}

The system exposes a RESTful API for communication between the frontend and backend. Endpoints are logically structured around resources, as shown in Table \ref{tab:api-endpoints}.
\begin{table}[htbp]
\centering
\caption{Key API Endpoint Specifications}
\label{tab:api-endpoints}
\resizebox{\textwidth}{!}{%
\begin{tabular}{l l p{6cm} l}
\toprule
\textbf{Method} & \textbf{Endpoint} & \textbf{Description} & \textbf{Auth Required} \\
\midrule
`POST` & `/auth/register` & Register a new user. & No \\
`POST` & `/auth/login` & Authenticate a user and return a JWT. & No \\
`POST` & `/api/rooms` & Create a new Quiz or Test Room. & Yes \\
`POST` & `/api/rooms/generate-ai` & Generate questions for a room using AI. & Yes \\
`POST` & `/api/v1/process-docx` & Upload a DOCX file for processing by the Python microservice. & Yes \\
`GET` & `/api/rooms/:roomCode` & Get public details for a specific room. & No \\
`POST` & `/api/rooms/:roomCode/submit` & Submit an answer to a question. & Yes \\
`POST` & `/api/rooms/:roomCode/log-proctor-event` & Log a suspicious event from a Test Room. & Yes \\
`GET` & `/api/discovery/nearby` & Find nearby rooms using GPS coordinates. & No \\
`POST` & `/api/vouchers/verify` & Verify a voucher via its QR code data. & Yes (Host) \\
\bottomrule
\end{tabular}%
}
\end{table}
\FloatBarrier
\subsection{Anti-Cheat System Design}
\label{subsec:anti-cheat-design}

Academic integrity in online testing presents unique challenges compared to traditional proctored exams. The TEKUTOKO platform implements a comprehensive, multi-layered anti-cheat system designed specifically for Test Rooms. This system balances the need for academic integrity with user privacy, system performance, and user experience constraints. The design philosophy follows a "deterrent-based" approach rather than attempting foolproof prevention, recognizing that lightweight browser-based monitoring cannot prevent all forms of cheating but can significantly reduce opportunistic violations.

\subsubsection{System Architecture Overview}

The anti-cheat system is architected as an integrated component of the Test Room frontend, with backend support for logging and analysis. The architecture consists of four primary layers:

\begin{enumerate}
    \item \textbf{Detection Layer:} Browser-based event listeners that monitor user behavior in real-time
    \item \textbf{Storage Layer:} Secure, encrypted client-side storage for violation counters and activity logs
    \item \textbf{Warning Layer:} User interface components that provide immediate feedback when violations are detected
    \item \textbf{Enforcement Layer:} Automated submission and blocking mechanisms triggered by violation thresholds
\end{enumerate}

The system operates primarily on the client side to minimize server load and enable real-time responsiveness, while critical data is synchronized to the server for permanent record-keeping and host review.

\subsubsection{Secure Data Storage Mechanism}

To prevent tampering with violation counters and activity logs stored in the browser, the system implements a custom \texttt{SecureStorage} class with multiple security layers:

\paragraph{Multi-Layer Encryption:}
Data stored in browser localStorage undergoes three stages of protection:

\begin{enumerate}
    \item \textbf{AES Encryption:} All data is encrypted using the AES-256 algorithm from the CryptoJS library with a device-specific secret key.
    
    \item \textbf{Obfuscation:} Encrypted data is further obfuscated using a Caesar cipher with variable shift values and Base64 encoding, making it unreadable in browser developer tools.
    
    \item \textbf{Integrity Verification:} Each data packet includes both a SHA-256 checksum and an HMAC-SHA256 signature to detect any tampering attempts.
\end{enumerate}

\paragraph{Device Fingerprinting:}
The encryption key is derived from a combination of browser and device characteristics, creating a unique fingerprint for each client:

\begin{lstlisting}[language=JavaScript, caption={Device Fingerprinting for Encryption Key Generation}]
generateSecretKey() {
  const canvas = this.getCanvasFingerprint();
  const screenInfo = `${window.screen.width}x${window.screen.height}`;
  const timezone = Intl.DateTimeFormat().resolvedOptions().timeZone;
  const language = navigator.language;
  const userAgent = navigator.userAgent.slice(0, 50);
  
  return CryptoJS.SHA256(
    `${canvas}-${screenInfo}-${timezone}-${language}-${userAgent}`
  ).toString();
}
\end{lstlisting}

The canvas fingerprinting technique exploits subtle differences in how browsers render graphics, creating a unique identifier that is extremely difficult to replicate. This prevents students from copying encrypted data between devices to circumvent violation tracking.

\paragraph{Data Packet Structure:}
Each stored item includes comprehensive metadata for validation:

\begin{lstlisting}[language=JavaScript, caption={Structure of Encrypted Data Packet}]
{
  data: { /* actual violation data */ },
  testId: "quiz_uuid_123",
  timestamp: 1701234567890,
  sessionId: "session_xyz_789",
  checksum: "sha256_hash_of_data",
  integrity: "hmac_signature"
}
\end{lstlisting}

On each retrieval, the system validates both the checksum and integrity signature. If either validation fails, the system flags a potential security violation and may terminate the test session to prevent cheating attempts through data manipulation.

\subsubsection{Behavior Detection Mechanisms}

The system monitors six categories of suspicious behavior using native browser APIs:

\paragraph{1. Tab Switching and Focus Loss Detection:}
Tab switching is one of the most common forms of digital cheating, allowing students to consult external resources. The system employs three complementary detection methods:

\begin{lstlisting}[language=JavaScript, caption={Multi-Method Tab Switch Detection}]
// Method 1: Visibility Change API
document.addEventListener('visibilitychange', () => {
  if (document.hidden && !isTestSubmitted) {
    logSuspiciousActivity('tabSwitches');
  }
});

// Method 2: Window Focus Events
window.addEventListener('blur', () => {
  if (!isTestSubmitted) {
    logSuspiciousActivity('tabSwitches');
  }
});

// Method 3: Page Unload Prevention
window.addEventListener('beforeunload', (e) => {
  if (!isTestSubmitted) {
    e.preventDefault();
    e.returnValue = '';
    logSuspiciousActivity('tabSwitches', 'Reload attempt');
  }
});
\end{lstlisting}

This multi-method approach ensures detection across different browsers and user actions (Alt+Tab, clicking outside the window, page refresh attempts).

\paragraph{2. Developer Tools Detection:}
Students may attempt to inspect or manipulate the DOM to view answers or modify quiz state. The system detects DevTools by monitoring changes in window dimensions:

\begin{lstlisting}[language=JavaScript, caption={DevTools Detection via Window Size Analysis}]
const detectDevTools = () => {
  const threshold = 160; // pixels
  const widthDiff = window.outerWidth - window.innerWidth;
  const heightDiff = window.outerHeight - window.innerHeight;
  
  if (widthDiff > threshold || heightDiff > threshold) {
    logSuspiciousActivity('devToolsAttempts');
  }
};

setInterval(detectDevTools, 1000); // Check every second
\end{lstlisting}

When DevTools is opened, it occupies space within the browser window, causing a measurable difference between the outer (browser) and inner (viewport) dimensions.

\paragraph{3. Copy/Paste Prevention:}
The system blocks content copying to prevent students from pasting questions into search engines:

\begin{lstlisting}[language=JavaScript, caption={Clipboard Event Blocking}]
['copy', 'cut', 'paste'].forEach(event => {
  document.addEventListener(event, (e) => {
    e.preventDefault();
    logSuspiciousActivity('copyAttempts');
    return false;
  });
});

// Prevent text selection
document.addEventListener('selectstart', (e) => {
  e.preventDefault();
  return false;
});
\end{lstlisting}

\paragraph{4. Context Menu and Keyboard Shortcut Blocking:}
Right-click menus and keyboard shortcuts provide access to browser features that could enable cheating:

\begin{lstlisting}[language=JavaScript, caption={Comprehensive Shortcut Blocking}]
// Block right-click menu
document.addEventListener('contextmenu', (e) => {
  e.preventDefault();
  logSuspiciousActivity('contextMenuAttempts');
});

// Block dangerous keyboard shortcuts
const handleKeyDown = (e) => {
  const blocked = (
    e.key === 'F12' || // DevTools
    (e.ctrlKey && e.shiftKey && 'IJC'.includes(e.key)) || // DevTools
    (e.ctrlKey && 'usacvx'.includes(e.key.toLowerCase())) // Save, Select, Copy, Paste
  );
  
  if (blocked) {
    e.preventDefault();
    logSuspiciousActivity('keyboardShortcuts', `Key: ${e.key}`);
  }
};
\end{lstlisting}

\paragraph{5. Screenshot Detection:}
While browser-based screenshot detection has limitations, the system logs PrintScreen key presses:

\begin{lstlisting}[language=JavaScript, caption={Screenshot Attempt Detection}]
document.addEventListener('keyup', (e) => {
  if (e.key === 'PrintScreen') {
    logSuspiciousActivity('screenshotAttempts');
  }
});
\end{lstlisting}

This does not prevent screenshots taken via operating system tools or external devices, but serves as a deterrent and provides evidence for review.

\paragraph{6. Fullscreen Enforcement:}
The system automatically enters fullscreen mode and monitors for exits:

\begin{lstlisting}[language=JavaScript, caption={Fullscreen Monitoring}]
// Enter fullscreen on test start
const enterFullscreen = async () => {
  await document.documentElement.requestFullscreen();
};

// Detect fullscreen exit
document.addEventListener('fullscreenchange', () => {
  const isFullscreen = !!document.fullscreenElement;
  if (!isFullscreen && testActive) {
    logSuspiciousActivity('tabSwitches', 'Exited fullscreen');
  }
});
\end{lstlisting}

Fullscreen mode reduces the ability to view other applications or reference materials simultaneously.

\subsubsection{Activity Logging and Violation Tracking}

Every detected suspicious event is logged with comprehensive metadata:

\begin{lstlisting}[language=JavaScript, caption={Structured Activity Logging}]
const logSuspiciousActivity = (type, details = '') => {
  const logEntry = {
    type: type,              // e.g., 'tabSwitches', 'devToolsAttempts'
    details: details,        // Additional context
    timestamp: new Date().toISOString(),
    questionIndex: currentQuestionIndex,
    sessionId: secureStorage.getSessionId()
  };
  
  // Add to in-memory log
  activityLog.push(logEntry);
  
  // Persist to encrypted storage
  secureStorage.setSecureItem(
    `activity_log_${testId}`, 
    activityLog, 
    testId
  );
  
  // Update violation counter
  suspiciousActivity[type]++;
  
  // Check threshold and potentially auto-submit
  checkViolationThreshold();
};
\end{lstlisting}

This granular logging enables post-hoc analysis by educators, providing context about when and where violations occurred during the test session.

\subsubsection{Progressive Warning and Enforcement}

The system uses a graduated response approach to violations:

\paragraph{Warning System:}
When a violation is detected, the system displays a prominent warning modal that briefly overlays the test content:

\begin{lstlisting}[language=JavaScript, caption={Warning Display Logic}]
const showWarning = (violationType) => {
  const messages = {
    tabSwitches: 'Tab switching detected. Stay on this page!',
    devToolsAttempts: 'Developer tools detected. Close them immediately!',
    copyAttempts: 'Copying is disabled during tests.',
    // ... other types
  };
  
  setWarningMessage(messages[violationType]);
  setShowWarning(true);
  
  // Auto-hide after 3 seconds
  setTimeout(() => setShowWarning(false), 3000);
};
\end{lstlisting}

The warning serves both as a deterrent (reminding students they are being monitored) and as a second chance (the test continues unless threshold is exceeded).

\paragraph{Automatic Submission and Blocking:}
When total violations reach a configurable threshold (default: 5), the test is automatically submitted and the student is blocked from continuing:

\begin{lstlisting}[language=JavaScript, caption={Violation Threshold Enforcement}]
const checkViolationThreshold = () => {
  const totalViolations = Object.values(suspiciousActivity)
    .reduce((sum, count) => sum + count, 0);
  
  if (totalViolations >= 5) {
    setBlockedForCheating(true);
    handleAutoSubmit('Excessive cheating behavior detected');
  }
};

const handleAutoSubmit = async (reason) => {
  const payload = {
    quiz_uuid: testId,
    student_username: username,
    answers: currentAnswers,
    cheating_detected: true,
    cheating_reason: reason,
    activity_log: activityLog,
    suspicious_activity: suspiciousActivity
  };
  
  await fetch('/api/v1/quiz/check-answers', {
    method: 'POST',
    body: JSON.stringify(payload)
  });
  
  // Clear storage and show blocked screen
  clearSecureStorage();
  setIsTestTerminated(true);
};
\end{lstlisting}

This automatic enforcement ensures that students cannot continue testing after repeated violations, while preserving evidence of their answers and behavior for educator review.

\subsubsection{Limitations and Design Trade-offs}

The anti-cheat system was designed with explicit acknowledgment of its limitations:

\begin{itemize}
    \item \textbf{Cannot detect off-screen activities:} The system cannot prevent students from using secondary devices (phones), consulting physical notes, or receiving assistance from others in the same room. These limitations are inherent to browser-based monitoring.
    
    \item \textbf{Potential false positives:} Legitimate technical issues (network problems, browser crashes, accidental key presses) may trigger violation logs. The system addresses this through graduated warnings rather than immediate penalties.
    
    \item \textbf{Privacy-preserving design:} Unlike camera-based proctoring, the system does not access webcams, microphones, or record keystrokes beyond specific blocked shortcuts. This protects student privacy but limits detection capabilities.
    
    \item \textbf{Deterministic detection only:} The system does not employ AI-based behavior analysis (e.g., detecting unusually fast answer patterns, identifying "too perfect" scores). Such analysis was considered but excluded due to complexity and false positive concerns.
\end{itemize}

These trade-offs reflect a deliberate design philosophy: the system is intended as a strong deterrent for opportunistic cheating in low- to medium-stakes assessments, not as a comprehensive solution for high-stakes exams requiring stronger guarantees.

\subsubsection{Integration with Results Reporting}

When a test is submitted (normally or automatically), all anti-cheat data is included in the results payload sent to the backend:

\begin{lstlisting}[language=JavaScript, caption={Anti-Cheat Data in Submission Payload}]
{
  quiz_uuid: "test_123",
  student_username: "student@example.com",
  answers: [ /* student answers */ ],
  
  // Anti-cheat metadata
  activity_log: [
    { type: 'tabSwitches', timestamp: '...', questionIndex: 3 },
    { type: 'devToolsAttempts', timestamp: '...', questionIndex: 7 }
  ],
  suspicious_activity: {
    tabSwitches: 2,
    devToolsAttempts: 1,
    copyAttempts: 0,
    // ...
  },
  cheating_detected: true,
  cheating_reason: "Exceeded violation threshold",
  security_violation_detected: false
}
\end{lstlisting}

This data is stored in the \texttt{proctoring\_logs} database table and made available to the host via the Results Dashboard, enabling informed decisions about test validity and potential academic integrity proceedings.

The anti-cheat system represents a pragmatic, privacy-conscious approach to maintaining academic integrity in online assessments. While not foolproof, its multi-layered design provides strong deterrence against common digital cheating methods while maintaining system performance and respecting student privacy. The comprehensive logging and graduated enforcement mechanisms give educators the tools to make informed judgments about test integrity.

\subsection{Security and Scalability Design}

\subsubsection{Authentication and Security}
\begin{itemize}
    \item \textbf{JWT-Based Authentication:} The system uses JSON Web Tokens for stateless authentication. After a successful login, the client receives an access token which is sent in the `Authorization` header of subsequent requests. The flow is visualized in Figure \ref{fig:jwt-flow}.
    \item \textbf{Password Security:} User passwords are never stored in plaintext. They are hashed using the `bcrypt` algorithm with a salt.
    \item \textbf{Voucher Security:} Each voucher is tied to a unique, cryptographically random string. The `is_redeemed` flag provides replay protection, ensuring a voucher can only be used once.
\end{itemize}

\begin{figure}[htbp]
\centering
\includegraphics[width=0.35\textwidth]{figures/jwt-flow.png}
\caption{JWT-Based Authentication and Authorization Flow}
\label{fig:jwt-flow}
\end{figure}

\FloatBarrier

\subsubsection{Scalability}
\begin{itemize}
    \item \textbf{Stateless Backend:} The Node.js API server is designed to be completely stateless. It does not store any session-specific data in memory, allowing requests from a single user to be distributed across any available server instance.
    \item \textbf{Database Scaling:} The MySQL database can be scaled vertically (more powerful hardware) or horizontally using read replicas to distribute read-heavy query loads.
    \item \textbf{Independent Microservice Scaling:} If DOCX parsing becomes a bottleneck, the Python microservice can be scaled independently by deploying more instances, without affecting the performance of the core Node.js application. This is crucial as document and image processing are CPU-intensive tasks.
\end{itemize}

\FloatBarrier

\section{Implementation Methodologies}
\label{sec:implementation}

This section details the technical implementation methodologies employed to translate the architectural designs into a functional system, covering the development environment, technology choices, and key implementation strategies.

\subsection{Development Environment and Technology Stack}

The technology stack was selected to optimize for developer productivity, performance, scalability, and a rich, modern user experience. The key technologies are summarized in Table \ref{tab:tech-stack}.

\renewcommand{\arraystretch}{1.5}
\setlength{\tabcolsep}{8pt}
\begin{longtable}{
>{\raggedright\arraybackslash}p{2.8cm} 
>{\centering\arraybackslash}p{3.2cm} 
>{\raggedright\arraybackslash}p{8cm}}

\caption{Technology Stack and Rationale} 
\label{tab:tech-stack} \\

\toprule
\textbf{Component} & \textbf{Technology / Library} & \textbf{Rationale} \\
\midrule
\endfirsthead

\multicolumn{3}{l}{\tablename\ \thetable{}: Technology Stack and Rationale \textit{(continued)}} \\[0.25em]
\toprule
\textbf{Component} & \textbf{Technology / Library} & \textbf{Rationale} \\
\midrule
\endhead

\midrule
\multicolumn{3}{r}{\textit{(continued on next page)}} \\ 
\endfoot

\bottomrule
\endlastfoot

\textbf{Frontend} & \textbf{React 18 (with Vite)} &
A high-performance library for building dynamic, component-based SPAs.
Vite provides a significantly faster development experience than traditional bundlers. \\[0.5em]

\textbf{Styling} & \textbf{Tailwind CSS} &
A utility-first CSS framework that enables rapid, consistent, and responsive UI development directly within the JSX markup. \\[0.5em]

\textbf{Backend (Core)} & \textbf{Node.js / Express.js} &
A lightweight and efficient JavaScript runtime, ideal for building fast, scalable, and I/O-intensive REST APIs and handling real-time connections. \\[0.5em]

\textbf{Backend (Microservice)} & \textbf{Python 3 / FastAPI} &
A high-performance and robust framework for building the specialized DOCX parsing microservice, leveraging Python's strong ecosystem for data processing and managing external processes. \\[0.5em]

\textbf{Database} & \textbf{MySQL} &
A reliable, widely-used relational database system that ensures data integrity and supports complex queries required for analytics and user data management. \\[0.5em]

\textbf{File Storage} & \textbf{Firebase Storage} &
A scalable and secure cloud storage solution that simplifies file uploads and management through its powerful SDKs and direct-to-cloud upload capabilities. \\[0.5em]

\textbf{AI Service} & \textbf{Google Gemini API} &
A state-of-the-art large language model used for the AI Question Generator, capable of understanding complex prompts and producing high-quality, structured JSON output. \\[0.5em]

\textbf{Deployment} & \textbf{Vercel, Docker} &
Vercel provides seamless CI/CD and optimized hosting for the React frontend.
Docker is used to containerize the backend services for consistent, portable deployment. \\[0.5em]

\end{longtable}

\subsection{Use Case Analysis}

\subsubsection{System Actors}
\begin{itemize}
    \item \textbf{Participant (User):} A registered user who joins rooms, participates in quizzes/tests, discovers content, and interacts with social features.
    \item \textbf{Host (Educator):} A user with elevated privileges to create, manage, and monitor game rooms and test rooms.
    \item \textbf{Administrator:} A privileged user responsible for overseeing the entire platform, managing content, and viewing system-wide analytics.
\end{itemize}

\subsubsection{Use Case Diagram}
Figure \ref{fig:use-case-diagram} illustrates the primary interactions between the actors and the TEKUTOKO system.

\begin{figure}[htbp]
\centering
\includegraphics[width=0.45\textwidth]{figures/use-case-diagram.png}
\caption{Use Case Diagram for the TEKUTOKO System}
\label{fig:use-case-diagram}
\end{figure}

\FloatBarrier

\subsection{Implementation Strategies}

Key implementation strategies employed include:

\begin{itemize}
    \item \textbf{Component-Based Development:} The React frontend was developed using a component-based architecture, promoting code reuse and maintainability.
    
    \item \textbf{API-First Design:} API endpoints were designed and documented before implementation, ensuring clear contracts between frontend and backend.
    
    \item \textbf{Containerization:} Both backend services were containerized using Docker, ensuring consistent deployment environments and facilitating scalability.
    
    \item \textbf{Continuous Integration:} Git-based version control with automated deployment pipelines ensured rapid iteration and reliable releases.
    
    \item \textbf{Human-in-the-Loop AI:} AI-generated content always requires human review before being made available to learners, ensuring quality and accuracy.
\end{itemize}

This comprehensive methodological approach, spanning requirements engineering, system design, and implementation, provides the foundation for a robust, scalable, and effective e-learning platform that addresses the research objectives outlined in Chapter \ref{chap:introduction}.
